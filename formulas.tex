\documentclass{article}

%Para reconocer tildes y otras cosas de español
\usepackage[spanish]{babel}
\usepackage[utf8]{inputenc}
\usepackage{euscript}
\usepackage{amsmath,amsfonts}
\usepackage{amssymb}
\usepackage{mathrsfs}
\usepackage{enumerate}

%Símbolos de uso común:
\newcommand{\RR}{\mathbb{R}}
\renewcommand{\Re}{\text{Re\,}}
\newcommand{\QQ}{\mathbb{Q}}
\newcommand{\NN}{\mathbb{N}}
\newcommand{\CC}{\mathbb{C}}
\newcommand{\ZZ}{\mathbb{Z}}
\renewcommand{\SS}{\EuScript{S}}
%Marco de Frechet-Serret
\newcommand{\vtan}{\mathbf{t}}
\newcommand{\nor}{\mathbf{n}}
\newcommand{\bin}{\mathbf{b}}

%Las ecuaciones llevan la seccion
\numberwithin{equation}{section}



%Norma de un vector
\renewcommand{\v}{\Vert}

\begin{document}
\section{Introducción}
Se imitará la notación del libro \cite{edg}, de antemano se asume que todas las curvas y superficies son suaves a menos que se diga lo contrario.

\section{Curvas}
\subsection{Curvatura}
La curvatura de una curva $\gamma$ parametrizada con respecto a su longitud de arco $s$ es la segunda derivada con respecto a $s$:
\begin{equation} \kappa = \v \ddot \gamma(s)\v \end{equation}
Si el parámetro de $\gamma$ no es necesariamente su longitud de arco, entonces tenemos:
\begin{equation} \kappa = \frac{\v \dot\gamma \times \ddot\gamma\v}{\v \dot \gamma\v^3}\end{equation}
Si es una curva plana $\mathbf{\gamma}\colon [a,b] \subset \mathbb{R} \to\mathbb{R}^2$ entonces esta definida por:
\begin{equation} \kappa = \frac{ |\dot x \ddot y- \ddot x \dot y|}{\{{\dot x}^2+{\dot y}^2\}^{3/2}}\end{equation}
Si la curva esta en coordenadas polares mediante $\rho = \rho(\theta)$ entonces la curvatura es:
\begin{equation} \kappa = \frac{ 2{\dot \rho}^2 - \rho{\ddot \rho}+{\rho}^2}{\{{\dot \rho}^2+{\rho}^2\}^{3/2}}\end{equation}

\subsection{Torsión}
La torsión $\tau$ de una curva $\gamma$ está definida por:
\begin{equation} \dot\bin = -\tau \nor \end{equation}
    Para una curva $\gamma$:
    \begin{equation}
        \tau = -\frac{(\dot\gamma \times \ddot\gamma)\cdot \dddot \gamma}{\v \dot\gamma\times\ddot\gamma \v^2}
    \end{equation}

\subsection{Las Ecuaciones de Frenet-Serret}
Si $\gamma$ esta arco-parametrizada o es de velocidad unitaria la curva en  $\mathbf{R}^3$ estara dada por:
\begin{equation}
\begin{aligned}
\dot\vtan &= \kappa \nor \\
\dot\nor &=  -\kappa \vtan + \tau \bin \\
\dot\bin &= -\tau \nor \\
\end{aligned}
\end{equation}

\section{Superficies}
\subsection{Formas Fundamentales}
Para una superficie $\sigma(u,v)$ la primera forma fundamental es $E\,du^2 + 2F\,du\,dv + G\,dv^2$ 
    \[E = \sigma_u\cdot \sigma_u, \quad 
    F = \sigma_u\cdot \sigma_v, \quad 
    G = \sigma_v\cdot \sigma_v \]

Y  la segunda formas fundamental es: $L\,du^2 + 2M\,du\,dv + N\,dv^2$ donde:
\begin{subequations}
\begin{align}
 L = \sigma_{uu} \cdot N &= -\sigma_u\cdot N_u \\
 M = \sigma_{uv} \cdot N &= -\sigma_u\cdot N_v = -\sigma_v\cdot N_u\\
    N = \sigma_{vv} \cdot N &= -\sigma_v\cdot N_v 
\end{align}
\end{subequations}

\begin{thebibliography}{9}
\bibitem{edg}
Andrew Pressley,
\emph{Elementary Differentiable Geometry},
Springer Undergraduate Mathematics Series (SUMS),
Second Edition, 
2010.

\end{thebibliography}
\end{document}
