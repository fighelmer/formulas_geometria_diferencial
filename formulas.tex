\documentclass{article}

%Para reconocer tildes y otras cosas de español
\usepackage[spanish]{babel}
\usepackage[utf8]{inputenc}
\usepackage{euscript}
\usepackage{amsmath,amsfonts}
\usepackage{amssymb}
\usepackage{mathrsfs}
\usepackage{enumerate}

%Símbolos de uso común:
\newcommand{\RR}{\mathbb{R}}
\renewcommand{\Re}{\text{Re\,}}
\newcommand{\QQ}{\mathbb{Q}}
\newcommand{\NN}{\mathbb{N}}
\newcommand{\CC}{\mathbb{C}}
\newcommand{\ZZ}{\mathbb{Z}}
\renewcommand{\SS}{\EuScript{S}}
%Marco de Frechet-Serret
\newcommand{\vtan}{\mathbf{t}}
\newcommand{\nor}{\mathbf{n}}
\newcommand{\bin}{\mathbf{b}}



%Norma de un vector
\renewcommand{\v}{\Vert}

\begin{document}
\section{Introducción}
Se imitará la notación del libro \cite{edg}, de antemano se asume que todas las curvas y superficies son suaves a menos que se diga lo contrario.

\section{Curvas}
\subsection{Curvatura}
La curvatura de una curva $\gamma$ parametrizada con respecto a su longitud de arco $s$ es la segunda derivada con respecto a $s$:
$$\kappa = \v \ddot \gamma(s)\v$$
Si el parámetro de $\gamma$ no es necesariamente su longitud de arco, entonces tenemos:
$$\kappa = \frac{\v \ddot\gamma \times \dot\gamma\v}{\v \dot \gamma\v}$$

\subsection{Torsión}
La torsión $\tau$ de una curva $\gamma$ está definida por:
\[ \dot\bin = -\tau \nor \] 

\section{Superficies}
\subsection{Formas Fundamentales}
Para una superficie $\sigma(u,v)$ las formas fundamentales son $E\,du^2 + 2F\,du\,dv + G\,dv^2$ y  $L\,du^2 + 2M\,du\,dv + N\,dv^2$ donde:
\begin{align}
    E = \sigma_u\cdot \sigma_u \quad & L = \sigma_{uu} \cdot N = -\sigma_u\cdot N_u \\
    F = \sigma_u\cdot \sigma_v \quad & M = \sigma_{uv} \cdot N = -\sigma_u\cdot N_v\\
    G = \sigma_v\cdot \sigma_v \quad & N = \sigma_{vv} \cdot N = -\sigma_v\cdot N_v
\end{align}


\begin{thebibliography}{9}
\bibitem{edg}
Andrew Pressley,
\emph{Elementary Differentiable Geometry},
Springer Undergraduate Mathematics Series (SUMS),
Second Edition, 
2010.

\end{thebibliography}
\end{document}
